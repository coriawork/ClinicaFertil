Categoría: Sujeto
Sujeto: Paciente
Noción: Mujer que busca realizar tratamientos de fertilidad para lograr un embarazo o preservar su fertilidad. Puede estar sola o en pareja y puede realizar múltiples tratamientos a lo largo del tiempo
Impacto: 
	• Solicita turnos médicos con médicos específicos o asignación aleatoria
	• Selecciona horario de un dia asignado por el médico para seguimiento
	• Visualiza y descarga órdenes necesarias
	• Proporciona datos personales completos durante el registro
	• El paciente Informa de antecedentes clínicos y ginecológicos al médico durante el turno
	• Proporciona antecedentes familiares describiendo patologías hereditaria
	• Se realiza estudios médicos solicitados por el médico tratante
	• Consulta resúmenes de su historial clínico
	• Firma consentimientos informados antes de iniciar medicación de estimulación ovárica
	• Decide sobre descarte de embriones no utilizados según sus preferencias personales



Categoría: Sujeto
Sujeto: Médico Tratante
Noción: Profesional médico especializado en fertilidad asignado específicamente a pacientes particulares. Tiene acceso limitado únicamente a los historiales de sus pacientes asignados.
Impacto: 
	• Accede y edita historiales clínicos de sus pacientes asignados
	• Registra antecedentes clínicos quirúrgicos mediante listas predefinidas específicas
	• Registra información sobre consumo de alcohol 
	• Registra antecedentes ginecológicos incluyendo menarca y características del ciclo menstrual
	• Registra información fenotípica
	• Realiza y Registra un examen físico
	• Usa calcula a través de un módulo el nivel de tabaquismo del paciente
	• Selecciona estudios médicos requeridos desde listas categorizadas predefinidas
	• Genera órdenes médicas tanto como para medicamento como para estudios y que se envían por correo electrónico al paciente
	• Pide y registra datos de la pareja del paciente (en caso de tenerla y ser necesario)
	• Analiza resultados de estudios determinando viabilidad de ovocitos y espermatozoides
	• Pedir compatibilidad con paciente al banco de semen
	• Visualiza y Utiliza herramienta de arbol genealogico para seguimiento de datos genéticos
	• Descarga planilla de consentimientos para firma antes de iniciar medicación
	• Registra el tratamiento de cada paciente con tipo, dosis, duración y droga
	• Programa calendarios de monitoreo con controles en días específicos durante estimulación
	• Registra observaciones de seguimiento durante monitoreo de estimulación ovárica
	• Programa punciones foliculares asignando día y hora específica
	• Administra información de los embriones
	• Consulta al paciente cuáles son sus planes y establece un objetivo
	• Registra datos sobre tratamientos (si hay beta positiva, embarazo, saco y nacimiento)
	• Cancela tratamientos

Categoría: Sujeto
Sujeto: Director Medico
Noción: Profesional médico con máximo nivel de autoridad en la clínica de fertilidad. Tiene acceso completo a todos los pacientes del sistema sin restricciones de asignación y a las estadísticas.
Impacto: 
	• Accede a todos los historiales clínicos de los pacientes
	• Accede y edita datos médicos de cualquier paciente
	• Consulta información de auditoría sobre modificaciones realizadas en historiales
	• Evalúa el desempeño individual de cada médico tratante mediante métricas específicas
	• Revisa estadísticas generales como tasas de embarazo del último año
	• Visualiza que medico edito los datos de algún paciente

Categoría: Sujeto
Sujeto: Operador de Laboratorio
Noción: Especialistas técnicos que realizan manipulaciones a nivel celular con gametos y embriones durante los tratamientos de fertilidad. Requieren acceso a datos médicos de todos los pacientes para realizar su trabajo correctamente sin restricciones de asignación médica.

Impacto: 
	• Registra información detallada de punciones foliculares incluyendo fecha, hora y número de quirófano
	• Documenta eventuales problemas ocurridos durante procedimientos quirúrgicos
	• Genera códigos únicos para cada ovocito extraído siguiendo formato específico establecido
	• Evalúa estado inicial de ovocitos clasificándolos como muy inmaduro, inmaduro o maduro
	• Ejecuta procesos de criopreservación para ovocitos maduros destinados a uso futuro
	• Mantiene trazabilidad completa del viaje de cada ovocito desde extracción hasta destino
	• Realiza fecundación mediante técnicas FIV como método estándar de laboratorio
	• Aplica técnica ICSI cuando FIV no funciona adecuadamente según protocolos establecidos
	• Genera identificadores únicos para cada embrión resultante del proceso de fecundación
	• Registra información completa de embriones incluyendo fecha, ovocito origen y técnica utilizada
	• Gestiona estudios PGT sobre embriones seleccionados registrando resultados como viable o no viable
	• Ejecuta procesos de criopreservación de embriones registrando ubicación en tubos de nitrógeno (tuvo + rack)
	• Accede a datos médicos de todos los pacientes necesarios para realizar manipulaciones celulares

Categoría: Sujeto
Sujeto: Admin
Noción: Usuario técnico responsable de la gestión general del sistema y configuración de entidades básicas.Es el responsable de crear usuarios médicos y mantener configuraciones generales.
Impacto: 
	• Crea usuarios médicos en el sistema asignando roles correspondientes

Categoría: Sujeto
Sujeto: Pareja Femenina
Noción: Mujer que acompaña a la paciente en tratamientos reproductivos. En método ROPA aporta ovocitos mientras la otra gesta. En otros casos no se registra.
Impacto: 
	• Completa información médica completa en método ROPA
	• Registra antecedentes ginecológicos cuando corresponde

Categoría: Sujeto
Sujeto: Pareja masculina
Noción: Hombre que acompaña a la paciente en tratamientos reproductivos aportando espermatozoides para fecundación. Completa información similar a la paciente adaptada a su condición.
Impacto: 
	• Proporciona datos personales completos, si el es el que va aportar espermatozoides
	• Se realiza estudios de semen cuando corresponde
